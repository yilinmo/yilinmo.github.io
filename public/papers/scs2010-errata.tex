\documentclass{article}

\usepackage{subfigure}
\usepackage{amssymb, amsmath, amsfonts}
\usepackage{amsthm}
\usepackage{tikz}
\usepackage{pgfplots}

\newtheorem{proposition}{Proposition}
\newtheorem{theorem}{Theorem}
\newtheorem{definition}{Definition}
\newtheorem{lemma}{Lemma}
\newtheorem{conjecture}{Conjecture}
\newtheorem{corollary}{Corollary}
\newtheorem{remark}{Remark}
\newtheorem{assumption}{Assumption}

\newlength\figureheight
\newlength\figurewidth

\newcommand{\tikzdir}[1]{tikz/#1.tikz}
\newcommand{\inputtikz}[1]{\tikzsetnextfilename{#1}\input{\tikzdir{#1}}}

\DeclareMathOperator*{\argmin}{arg\; min}     % argmin
\DeclareMathOperator*{\argmax}{arg\; max}     % argmax
\DeclareMathOperator*{\tr}{tr}     % trace
\DeclareMathOperator{\Cov}{Cov}
\DeclareMathOperator{\logdet}{log\;det}
\DeclareMathOperator{\vecspan}{span}

\title{Errata on Theorem 2 in ``False Data Inject Attacks in Control Systems''}

\author{Yilin Mo}

\begin{document} \maketitle
The proof can be divided into the following steps:
\begin{enumerate}
\item $A$ matrix must be unstable. Suppose otherwise. Then we know there exists an $l$, such that $\|A^l\| = \alpha < 1$. As a result
  \begin{align*}
   \|\Delta e_{k+l}\| = \|A^l\Delta e_k - A^{l-1}K\Delta z_{k+1} -\cdots - K \Delta z_{k+l}\| \leq \alpha \|\Delta e_k\| + \|A^{l-1}K\| + \cdots + \|K\|.
  \end{align*}
  As a result, $\|\Delta e_k\|$ will be bounded, which contradicts with the fact that $\|\Delta e_k\|  \rightarrow \infty$.
\item Given an unstable $A$, there exists an $l > n-1$, such that if $\|A^lp\| \geq \|p\|$, then $\lim_{k\rightarrow\infty} A^kp \neq 0$.

  Let us decompose $A$ as
  \begin{align*}
    V \begin{bmatrix}
      A_u&0\\
      0&A_s
    \end{bmatrix}V^{-1},
  \end{align*}
 where $A_s$ is strictly stable and $A_u$ contains all the unstable and critically stable eigenvalues. Define
  \begin{align*}
    \tilde A \triangleq  V \begin{bmatrix}
      0&0\\
      0&A_s
    \end{bmatrix}V^{-1}.
  \end{align*}
Since $\tilde A$ is stable, there exists an $l > n-1$, such that $\|\tilde A^l\|  <1$. 

Now suppose that 
\begin{align*}
  V^{-1}p = \begin{bmatrix}
    p_u\\
    p_s
  \end{bmatrix}.
\end{align*}
If $p_u = 0$, then
\begin{align*}
  A^lp = V \begin{bmatrix}
    A_u^lp_u\\
    A_s^lp_s
  \end{bmatrix} = V \begin{bmatrix}
    0\\
    A_s^lp_s
  \end{bmatrix}  = \tilde A^l p.
\end{align*}
As a result,
\begin{align*}
  \|A^lp \| = \|\tilde A^l p\| \leq \|\tilde A^l\|\times \|p\| < \|p\|,
\end{align*}
Thus, if $\|A^lp\| \geq \|p\|$, we can conclude that $p_u$ cannot be zero, which further implies that
\begin{align*}
 \lim_{k\rightarrow\infty} A^k p = V \begin{bmatrix}
    A_u^lp_u\\
    0
  \end{bmatrix}\neq 0.
\end{align*}
\item We can now conclude, using Lemma 2, that if $\|A^lp\| \geq \|p\|$, then there exists an unstable eigenvector $v$, such that
  \begin{align*}
    v \in \vecspan\left\{p,Ap,\cdots,A^lp\right\}
  \end{align*}
\item The proof of Theorem 2 will remain almost the same. The difference is that  we need to ensure $\{p_{j_k}\}$, ..., $\{p_{j_k-l}\}$ all converge, instead of $\{p_{j_k}\}$, ..., $\{p_{j_k-n+1}\}$ all converge (in the paragraph after Equation~(31)).

  The next difference is the first paragraph of page 7. Instead of $\|Aq_1\| \geq \|q_1\|$, we have $\|A^l{q_l}\| \geq \|q_l\|$.
\end{enumerate}
\end{document}

%%% Local Variables:
%%% TeX-command-default: "Latexmk"
%%% End: